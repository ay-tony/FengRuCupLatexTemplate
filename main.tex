\documentclass{buaa_fengru}

\graphicspath{{fig/}} % 图片引用路径

\title{第三十四届“冯如杯”竞赛创意赛道论文\\撰写格式规范} % 标题
\subtitle{---以 \LaTeX 上的应用为例} % 副标题设置(如果无可以删去)

\abstractzh{} % 中文摘要内容
\wordszh{} % 中文关键词

\abstracten{} % 英文摘要内容(如果无可以删去)
\wordsen{} % 英文关键词(如果无可以删去)

\begin{document}

\maketitle % 打印封面和摘要页

% 以下正文

\section{论文组成}

包括 5 个部分,顺序依次为:

\begin{enumerate}[nosep]
    \item 封面(中文)
    \item 中文摘要、关键词(中文、英文)
    \item 主体部分
    \item 结论
    \item 参考文献
\end{enumerate}

\section{论文的书写规范}

论文一律由在计算机上输入、排版、定稿后转成PDF格式,在集中申报时通过网络上传。

\textcolor{red}{
    \bfseries
    特别注意:封面及全文中不能出现作者姓名、学院、专业、指导老师的相关信息。
}

\subsection{字体和字号}

\begin{description}[nosep]
    \item[论文题目] 二号,华文中宋体加粗,居中
    \item[副标题] 三号,华文新魏,居右(可省略)
    \item[章标题] 三号,黑体,居中
    \item[节标题] 四号,黑体,居左
    \item[条标题] 小四号,黑体,居左
    \item[正文] 小四号,中文字体为宋体,西文字体为 Times New Roman体,首行缩进,两端对齐
    \item[页码] 五号 Times New Roman 体,数字和字母
\end{description}

\subsection{页边距及行距}

学术论文的上边距:25mm;下边距:25mm;左边距:30mm;右边距 20mm。

章、节、条三级标题为单倍行距,段前、段后各设为 0.5 行(即前后各空 0.5 行)。

正文为 1.5 倍行距,段前、段后无空行(即空 0 行)。

\subsection{页眉}

页眉内容为\textbf{北京航空航天大学第三十四届“冯如杯”竞赛创意赛道参赛作品},内容居中。

页眉用小五号宋体字,页眉标注从论文主体部分开始(引言或第一章)。

\textcolor{red}{
    \bfseries
    请注意论文封面无页眉。
}

\subsection{页码}

论文页码从“主体部分(引言、正文、结论)”开始,直至“参考文献”结束,用五号阿拉伯数字连续编码,页码位于页脚居中。\textcolor{red}{\bfseries 封面、题名页不编页码。}

摘要、目录、图标清单、主要符号表用五号小罗马数字连续编码,页码位于页脚居中。

\subsection{图、表及其附注}

图和表应安排在正文中第1次提及该图、表的文字的下方,当图或表不能安排在该页时,应安排在该页的下一页。

\subsubsection{图}

图题应明确简短,用\textbf{五号宋体加粗},数字和字母为\textbf{五号 Times New Roman 体加粗},图的编号与图题之间应空半角2格。图的编号与图题应置于图下方的居中位置。图内文字为\textbf{5 号宋体},数字和字母为\textbf{5 号 Times New Roman} 体。曲线图的纵横坐标必须标注“量、标准规定符号、单位”,此三者只有在不必要注明(如无量纲等)的情况下方可省略。坐标上标注的量的符号和缩略词必须与正文中一致。

\subsubsection{表}

表的标号应采用从1开始的阿拉伯数字编号,如:“表 1”、“表 2”、……。表编号应一直连续到附录之前,并与章、节和图的编号无关。只有一幅表,仍应标为“表 1”。表题应明确简短,用\textbf{五号宋体加粗},数字和字母为\textbf{五号 Times New Roman 体加粗},表的编号与表题之间应空半角 2 格。表的编号与表头应置于表上方的居中位置。表内文字为 \textbf{5 号宋体},数字和字母为 \textbf{5 号 Times New Roman 体}。  

\subsubsection{附注}

图、表中若有附注时,附注各项的序号一律用“附注+阿拉伯数字+冒号”,如:“附注1:”

附注写在图、表的下方,一般采用5号宋体。

\section{论文每部分内容的具体要求}

\subsection{论文的封面}

论文题目:应准确、鲜明、简洁,能概括整个论文中最主要和最重要的内容。题目\textbf{不超过 20 个中文字},若语意未尽,可用副标题补充说明。副标题应处于从属地位,一般可在题目的下一行用破折号“——”引出。论文题目应避免使用不常用缩略词、首字母缩写字、字符、代号和公式等。

\subsection{摘要}

摘要内容包括:“摘要”字样,摘要正文,关键词。在摘要的最下方另起一行,用显著的字符注明文本的关键词。

摘要是论文内容的简短陈述,应体现论文工作的核心思想。摘要一般约 500 字。摘要内容应涉及本项科研工作的目的和意义、研究思想和方法、研究成果和结论。

关键词是为用户查找文献,从文中选取出来用来揭示全文主题内容的一组词语或术语,应尽量采用词表中的规范词(参照相应的技术术语标准)。关键词一般为3--8个,按词条的外延层次排列。关键词之间用逗号分开,最后一个关键词后不打标点符号。

\subsection{主体部分}

论文主体一般应包括:引言(或绪论)、正文等部分。

\subsubsection{引言}

引言部分需包括如下内容:

\begin{enumerate}[nosep]
    \item 阐述作品背景、创意来源;
    \item 介绍并分析该领域国内外研究现状或解决方案,及其优缺点。
\end{enumerate}

\subsubsection{正文}

正文部分需包括如下内容:

\begin{enumerate}[nosep]
    \item 作品核心创意,包含创意产生过程,核心思路描述等;
    \item 创意可行性分析,包含技术实现思路,相关技术分析,预计技术难点等;
    \item 作品或创意应用前景,包含应用场景,市场需求和推广模式等。
\end{enumerate}

论文正文部分需分章节撰写,每章应另起一行。

章节标题要突出重点,简明扼要、层次清晰。字数一般在15字以内,不得使用标点符号。标题中尽量不采用英文缩写词,对必须采用者,应使用本行业的通用缩写词。层次以少为宜,根据实际需要选择。三级标题的层次按章(如“一、”)、节(如 “(一)”)、条(如“1.”)的格式编写,各章题序的阿拉伯数字用 Times New Roman 体。

\subsection{结论}

论文的结论单独作为一章,但不加章号。


\subsection{参考文献}

凡有直接引用他人成果(文字、数字、事实以及转述他人的观点)之处,均应加标注说明列于参考文献中,以避免论文抄袭现象的发生。

具体范例如下。

\subsubsection{著作}

[序号]作者.书名[标识码].出版地:出版社,出版年.

[1]张志建.严复思想研究[M].桂林:广西师范大学出版社,1989.

[2]马克思恩格斯全集(第 1 卷)[M].北京:人民出版社,1956.

说明:马克思恩格斯全集、毛选、邓选以及《鲁迅全集》、《朱光潜全集》等每一卷设一个序号。

\subsubsection{译著}

[序号]国名或地区(用圆括号)原作者.书名[标识码].译者.出版地:出版社,出版年.

[1](英)霭理士.性心理学[M].潘光旦译.北京:商务印务馆,1997.

\subsubsection{古典文献}

文史古籍类引文后加序号,再加圆括号,内加注书名、篇名或页码。例如:文中“……孔子独立郭东门。”[1](《史记•孔子世家》)

\subsubsection{论文集}

[序号]编者.书名[标识码].出版地:出版社,出版年.

[1]伍蠡甫.西方论文选(下册)[C].上海:上海译文出版社,1979.论文集中特别标出其中某一文献

[序号]其中某一文献的著者.某一文献题名[A].论文集编者.论文集题名[C].出版地:出版单位,出版年.

[1]别林斯基.论俄国中篇小说和果戈理君的中篇小说[A].伍蠡甫.西方文论选:下册[C].上海:上海译文出版社,1979.

\subsubsection{期刊文章}

[序号]作者.篇名[标识码].刊名,年,(期).

[1]叶朗.《红楼梦》的意蕴[J].北京大学学报(哲学社会科学版),1989,(2)

\subsubsection{报纸文章}

[序号]作者.篇名[标识码].报纸名,出版日期(版次)

[1]谢希德.创造学习的新思路[N].人民日报,1998-12-25(10)

\subsubsection{外文文献}

要求外文文献所表达的信息和中文文献一样多,但文献类型标识码可以不标出。

[1]Mansfeld,R.S.Busse. T.V. The Psychology of creativity and discovery,Chinago:NelsonHall,1981

[2]Setrnberg,R.T. The nature of creativity,New York:Cambridge University Press,1988

[3]Yong,L.S. Managing creative people. Journal of Create Behavior,1994,28(1)

说明:

\begin{enumerate}[nosep]
    \item 外文文献一定要用外文原文,切忌用中文叙述外文,如“牛津大学出版社,某某书,多少页”等等。
    \item 英文书名、杂志名用斜体,或画线标出。
\end{enumerate}

\subsection{正文中标注格式}

标注格式:引用参考文献标注方式应全文统一,标注的格式为[序号],放在引文或转述观点的最后一个句号之前,所引文献序号用\textbf{小 4 号 Times New Roman 体}、以上角标形式置于方括号中,如“……成果”$^{\text{[i]}}$。

\section{文件大小不超过 5M}

\section{模板使用指导}

\subsection{关于编译引擎}

请使用 XeLaTeX 编译本模板。

\subsection{关于英文摘要页}

模板提供英文摘要页. 如果在源文件中利用 \verb|\abstracten{}| 命令编辑英文摘要,就会自动生成英文摘要页。如果留空或不使用此命令,则不会生成英文摘要页。

\subsection{插入不编号标题}

格式中要求部分标题要求无编号,可以按照以下示例插入.

\begin{lstlisting}[language=Tex]
\section*{结论} % 插入无编号标题
\addcontentsline{toc}{section}{结论} % 在目录中插入对应链接
\end{lstlisting}

\subsection{插入引用}

模板中提供 \verb|\upcite{}| 命令以插入 Bibtex 引用,可以按照以下示例插入.

\begin{lstlisting}[language=Tex]
此处引用陈纪修老师的《数学分析》\upcite{陈纪修_2019_4477454}。
也可以引用多处文献\upcite{陈纪修_2019_4477454,谢惠民_2018_4477464}。
\end{lstlisting}

效果如下:

此处引用陈纪修老师的《数学分析》\upcite{陈纪修_2019_4477454}。
也可以引用多处文献\upcite{陈纪修_2019_4477454,谢惠民_2018_4477464}。

% 以下参考文献
\newpage
\bibliography{main} % 打印参考文献
\addcontentsline{toc}{section}{参考文献} % 向目录中添加插入参考文献行

\end{document}
